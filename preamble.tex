%%%%%%%%%%%%%%%%%%%%%%%%%%%%%%%%%%%%%%%%%%%%%%%%%%%%
%%%%%%%%%%%%%%%%%%%   Preamble   %%%%%%%%%%%%%%%%%%%
%%%%%%%%%%%%%%%%%%%%%%%%%%%%%%%%%%%%%%%%%%%%%%%%%%%%

\documentclass[titlepage,11pt,a4paper]{article}

\usepackage[english]{babel}             % Language german; replace ngerman by english when required
\usepackage[T1]{fontenc}                % For better wort separation
\usepackage[utf8]{inputenc}             % For äöüß etc.
\usepackage{enumerate}					% Better enumerate definition
\usepackage{amsthm}                     % For theorems, definitions, propositions etc.
\usepackage{amsmath}                    % Makes everything better
\usepackage{amssymb}                    % e. g. for \mathbb or \mathfrak
\usepackage{mathtools}                  % For aligning the columns of a matrix (pmatrix*)
\usepackage{fancyhdr}					% Fancy Header
\usepackage{listings}                   % For writing code
\usepackage{tikz}                       % Tikz ist kein Zeichenprogramm
\usepackage{svg}                        % For including svg files
\usepackage{babelbib}                   % Bibliography
\bibliographystyle{unsrt}               % also bib, may change later
\usepackage{stmaryrd}					% For \longarrownot\longrightarrow
\usepackage{hyperref}

%%%%%%%%%%%%%         Header         %%%%%%%%%%%%%%

\pagestyle{fancy}
\addtolength{\headheight}{\baselineskip}
\fancyhead[R]{\textsl{\thepage}}
\fancyhead[L]{\textsl{\leftmark}}
%\fancyhead[LE,RO]{\textsl{\thepage}}
%\fancyhead[LO,RE]{\textsl{\leftmark}}
\renewcommand{\headrulewidth}{0.4pt}

%%%%%%%%%%%%%         Footer         %%%%%%%%%%%%%%

%\fancyfoot[C]{\thepage/\pageref{LastPage}}
\fancyfoot[C]{}


%%%%%%%%%%%  Set code font to \texttt  %%%%%%%%%%%%

\lstset{
	basicstyle=\ttfamily,
	columns=fullflexible,
}


%%%%%%%%%%%%%    Special Commands    %%%%%%%%%%%%%%

\newtheoremstyle{break}%
	{}{}%
	{\itshape}{}%
	{\bfseries}{}% % Note that final punctuation is omitted.
	{\newline}{}

\theoremstyle{definition}
\newtheorem{definition}{Definition}[section]
\newtheorem{example}[definition]{Example}

\theoremstyle{plain}
\newtheorem{corollary}[definition]{Corollary}
\newtheorem{theorem}[definition]{Theorem}
\newtheorem{proposition}[definition]{Proposition}
\newtheorem{lemma}[definition]{Lemma}

\theoremstyle{remark}
\newtheorem{remark}[definition]{Remark}


\theoremstyle{break}
\newtheorem{breakcorollary}[definition]{Corollary}
\newtheorem{breaktheorem}[definition]{Theorem}
\newtheorem{breakproposition}[definition]{Proposition}
\newtheorem{breaklemma}[definition]{Lemma}




% Some shortcuts
\newcommand{\Nat}{\mathbb{N}}
\newcommand{\Integ}{\mathbb{Z}}
\newcommand{\Real}{\mathbb{R}}
\newcommand{\Rat}{\mathbb{Q}}
\newcommand{\Comp}{\mathbb{C}}
\newcommand{\terms}{\mathbb{T}}
\newcommand{\im}{\operatorname{Im}}     	
\newcommand{\Ker}{\operatorname{Ker}}
\newcommand{\Hom}{\operatorname{Hom}}   	
\newcommand{\ri}{\operatorname{ri}}
\newcommand{\Supp}{\operatorname{Supp}} 	
\newcommand{\Ann}{\operatorname{Ann}}
\newcommand{\HF}{\operatorname{HF}}     	
\newcommand{\HP}{\operatorname{HP}}
\newcommand{\HS}{\operatorname{HS}}    	
\newcommand{\HN}{\operatorname{HN}}
\newcommand{\hn}{\operatorname{hn}}     	
\newcommand{\mult}{\operatorname{mult}}
\newcommand{\LT}{\operatorname{LT}}     	
\newcommand{\LM}{\operatorname{LM}}	
\newcommand{\IG}{\mathcal{I}_G}
\newcommand{\MS}{\operatorname{MS}}     	
\newcommand{\LC}{\operatorname{LC}}
\newcommand{\GL}{\operatorname{GL}}     	
\newcommand{\NR}{\operatorname{NR}}
\newcommand{\id}{\operatorname{id}}		
\newcommand{\End}{\operatorname{End}}
\newcommand{\ord}{\operatorname{ord}}	
\newcommand{\rey}{\varrho}
\newcommand{\calA}{{\mathcal{A}}}
\newcommand{\isom}{\cong}
\newcommand{\calB}{{\mathcal{B}}}
\newcommand{\calC}{{\mathcal{C}}}
\newcommand{\calI}{{\mathcal{I}}}
\newcommand{\Rel}{{\operatorname{Rel}}}
\newcommand{\Mat}{{\operatorname{Mat}}}
\newcommand{\tr}{^{\rm tr}} % transposed matrix
\newcommand{\trace}{{\rm trace}}
\renewcommand{\det}{{\rm det}}
\newcommand{\characteristic}{{\rm char}}
%\newcommand{\rewrite}[1]{{\,\stackrel{#1}{\longrightarrow}_{\rm s}\,}}
%\newcommand{\nrewrite}[1]{{\,\stackrel{#1}{\longarrownot\longrightarrow}_{\rm s}\,}}
%\newcommand{\rewriteequiv}[1]{{\,\stackrel{#1}{\longleftrightarrow}_{\rm s}\,}}
%\newcommand{\reductionstep}[1]{{\,\stackrel{#1}{\longrightarrow}_{\rm ss}\,}}
\newcommand{\rewrite}[1]{{\,{\xrightarrow{\mathmakebox[0.5cm]{#1}}}_{\rm s}\,}}
\newcommand{\nrewrite}[1]{{\,{\longarrownot\xrightarrow{\mathmakebox[0.5cm]{#1}}}_{\rm s}\,}}
\newcommand{\rewriteequiv}[1]{{\,{\xleftrightarrow{\mathmakebox[0.5cm]{#1}}}_{\rm s}\,}}
\newcommand{\reductionstep}[1]{{\,{\xrightarrow{\mathmakebox[0.5cm]{#1}}}_{\rm ss}\,}}


% Inserts an empty page
\newcommand{\emptypage}{\newpage\null\thispagestyle{empty}\newpage}

% For restricting a function to a certain subset - f|_A
\newcommand\restr[2]{{% we make the whole thing an ordinary symbol
		\left.\kern-\nulldelimiterspace % automatically resize the bar with \right
		#1 % the function
		\vphantom{\big|} % pretend it's a little taller at normal size
		\right|_{#2} % this is the delimiter
}}

% Hyperlinks in TOC, references and citations
\hypersetup{
	colorlinks,
	citecolor=black,
	filecolor=black,
	linkcolor=black,
	urlcolor=black
}
