%%%%%%%%%%%%%%%%%%%%%%%%%%%%%%%%%%%%%%%%%%%%%%%%%%%%
%%%%%%%%%%%%%%%%%%%   Preamble   %%%%%%%%%%%%%%%%%%%
%%%%%%%%%%%%%%%%%%%%%%%%%%%%%%%%%%%%%%%%%%%%%%%%%%%%

\documentclass[titlepage,11pt,a4paper]{article} % add twoside to optionals if thesis is printed two-sided

\usepackage[utf8]{inputenc}             % For äöüß etc.
\usepackage{csquotes}                   % Recommended for babel package
\usepackage[english]{babel}             % Language english; replace by ngerman when required
\usepackage[T1]{fontenc}                % For better wort separation
\usepackage{lmodern}                    % Vector font
\usepackage{enumerate}                  % Better enumerate definition
\usepackage{amsthm}                     % For theorems, definitions, propositions etc.
\usepackage{amsmath}                    % Makes everything better
\usepackage{amssymb}                    % e. g. for \mathbb or \mathfrak
\usepackage{mathtools}                  % For aligning the columns of a matrix (pmatrix*)
\usepackage{fancyhdr}                   % Fancy Header
\usepackage{listings}                   % For writing code
\usepackage{tikz}                       % Tikz ist kein Zeichenprogramm
\usepackage{svg}                        % For including svg files
\usepackage{stmaryrd}                   % For \longarrownot\longrightarrow
\usepackage{hyperref}                   % For interlinks in references
\usepackage{relsize}                    % For \mathlarger
\usepackage{emptypage}                  % makes \pagestyle{empty} if \cleardoubleside skips one page
\usepackage{comment}
\usepackage{xspace}                     % use \xspace at end of newcommand to avoid "CNFbla" when typing \cnf bla
\usepackage{array}                      % For fixed column width in notations.tex

%%%%%%%%%%%%%         Header         %%%%%%%%%%%%%%

\pagestyle{fancy}
\addtolength{\headheight}{\baselineskip}
\fancyhead[LE,RO]{\textsl{\thepage}}
\fancyhead[RE]{\textsl{\rightmark}}
\fancyhead[LO]{\textsl{\leftmark}}
\renewcommand{\headrulewidth}{0.4pt}


%%%%%%%%%%%%%         Footer         %%%%%%%%%%%%%%

\fancyfoot[C]{}


%%%%%%%%%%%%% Bibliography      %%%%%%%%%%%%%%

\usepackage[backend=biber]{biblatex}
\addbibresource{literatur.bib}
\usepackage{csquotes} % Loading package recommended
\renewcommand*{\mkbibnamefamily}[1]{\textsc{#1}} % Print authors in smallcaps
\DeclareFieldFormat[article, inbook]{title}{#1} % no quotation marks around titles




%%%%%%%%%%%    Peter's CoCoA listing    %%%%%%%%%%%

\definecolor{altorange}{HTML}{F29300}
\definecolor{codeblue}{rgb}{0.13,0.13,1}
\definecolor{codegreen}{rgb}{0,0.6,0}
\definecolor{codegray}{rgb}{0.5,0.5,0.5}
\definecolor{codepurple}{rgb}{0.58,0,0.82}
\definecolor{backcolour}{rgb}{0.95, 0.95, 0.92}

\lstdefinelanguage{CoCoA}{
  morekeywords={Alias, Block, Break, Ciao, Continue, Define, Describe, Do, Elif, Else, End, EndBlock, EndTry, EndDefine, EndFor, EndForeach, EndIf, EndPackage, EndRepeat, EndUsing, EndWhile, Eof, False, For, Foreach, Global, Help, If, In, IsIn, NewLine, Not, On, Or, Package, Print, PrintLn, Quit, Repeat, Record, Return, Skip, Source, Step, Then, Time, To, True, Unset, Until, use, Using, Var, While, QQ, ZZ, BOOL, DegLex, DegRevLex, DEVICE, ERROR, FUNCTION, IDEAL, INT, Lex, MAT, MODULE, NULL, Null, PANEL, POLY, PosTo, RAT, RATFUN, RING, STRING, TAGGED, ToPos, TYPE, MODULEELEM, Xel, ZMOD},
  sensitive=false,
  morecomment=[l]{//},
  morecomment=[s]{/*}{*/},
  morestring=[b]"
}
  
\lstset{
  backgroundcolor=\color{backcolour},
  commentstyle=\color{codegray},
  keywordstyle=\color{codeblue},
  numberstyle=\tiny\color{codegray},
  stringstyle=\color{codegreen},
  basicstyle=\ttfamily\fontsize{9}{12}\selectfont{},
  breakatwhitespace=false,
  breaklines=true,
  captionpos=b,
  columns=fullflexible,
  keepspaces=true,
  language=CoCoA,
  numbers=left,
  numbersep=5pt,
  showspaces=false,
  showstringspaces=false,
  showtabs=false,
  tabsize=2,
  xleftmargin=1cm,
  xrightmargin=8mm,
  literate={~} {$\sim$}{1},
  literate={*}{{\char42}}1 {-}{{\char45}}1 {^}{{\char94}}1,
}


%%%%%%%%%%%%%    Special Commands    %%%%%%%%%%%%%%

\theoremstyle{definition}
\newtheorem{definition}{Definition}[section]
\newtheorem{example}[definition]{Example}
\newtheorem{remark}[definition]{Remark}

\theoremstyle{plain}
\newtheorem{corollary}[definition]{Corollary}
\newtheorem{theorem}[definition]{Theorem}
\newtheorem{proposition}[definition]{Proposition}
\newtheorem{lemma}[definition]{Lemma}





% Some shortcuts
\DeclareMathOperator{\im}{Im}
\DeclareMathOperator{\ri}{ri}
\DeclareMathOperator{\HF}{HF}
\DeclareMathOperator{\HP}{HP}
\DeclareMathOperator{\HS}{HS}
\DeclareMathOperator{\HN}{HN}
\DeclareMathOperator{\hn}{hn}
\DeclareMathOperator{\LT}{LT}
\DeclareMathOperator{\LM}{LM}
\DeclareMathOperator{\MS}{MS}
\DeclareMathOperator{\LC}{LC}
\DeclareMathOperator{\GL}{GL}
\DeclareMathOperator{\NR}{NR}
\DeclareMathOperator{\id}{id}
\DeclareMathOperator{\End}{End}
\DeclareMathOperator{\ord}{ord}
\DeclareMathOperator{\Ann}{Ann}
\DeclareMathOperator{\Ker}{Ker}
\DeclareMathOperator{\Hom}{Hom}
\DeclareMathOperator{\Rel}{Rel}
\DeclareMathOperator{\Mat}{Mat}
\DeclareMathOperator{\mult}{mult}
\DeclareMathOperator{\Supp}{Supp}
\DeclareMathOperator{\trace}{trace}
\DeclareMathOperator{\Char}{char} % \char already exists
\DeclareMathOperator{\tr}{tr} % transposed matrix
\DeclareMathOperator{\rank}{rk}
\DeclareMathOperator{\Lits}{Lits}
\DeclareMathOperator{\eval}{eval}
\DeclareMathOperator{\Pot}{\mathfrak{P}}
\DeclareMathOperator{\Nat}{\mathbb{N}}
\DeclareMathOperator{\Integ}{\mathbb{Z}}
\DeclareMathOperator{\Real}{\mathbb{R}}
\DeclareMathOperator{\Rat}{\mathbb{Q}}
\DeclareMathOperator{\Comp}{\mathbb{C}}
\DeclareMathOperator{\terms}{\mathbb{T}}
\DeclareMathOperator{\field}{\mathbb{F}}
\DeclareMathOperator{\calA}{\mathcal{A}}
\DeclareMathOperator{\calB}{\mathcal{B}}
\DeclareMathOperator{\calC}{\mathcal{C}}
\DeclareMathOperator{\calI}{\mathcal{I}}
\DeclareMathOperator{\calX}{\mathcal{X}}
\DeclareMathOperator{\calY}{\mathcal{Y}}
\DeclareMathOperator{\calZ}{\mathcal{Z}}
\newcommand{\lxor}{\mathop{\raisebox{0.5pt}{$\oplus$}}{}}
\DeclareMathOperator*{\LXOR}{\bigoplus}
\DeclareMathOperator*{\LAND}{\bigwedge}
\DeclareMathOperator*{\LOR}{\bigvee}
\newcommand{\liff}{\leftrightarrow}
\newcommand{\limplies}{\rightarrow}
\newcommand{\limpliedby}{\leftarrow}
\newcommand{\dotcup}{\mathop{\dot\cup}}
\newcommand{\true}{\mathcal{T}}
\newcommand{\True}{\true}
\newcommand{\false}{\mathcal{F}}
\newcommand{\False}{\false}
\newcommand{\Sols}{\mathcal{S}}
\newcommand{\Zeros}{\mathcal{Z}}
\newcommand{\cpp}{\textsf{C}\texttt{++}\xspace}
\newcommand{\bbar}[1]{{\mkern 4.0mu\overline{\mkern-4.0mu #1 \mkern-0.5mu}\mkern 0.5mu}}
\newcommand{\textmk}[1]{\textbf{#1}}  % marked expression e.g. in a definition
\newcommand{\textmkP}[1]{\textsl{#1}} % marked expression before its definition
\newcommand{\textmkI}[1]{\textsl{#1}} % marked expression during definition in introduction
\newcommand{\textmkN}[1]{\textsf{#1}} % for names like ApCoCoA or CrytoMiniSat
\newcommand{\code}[1]{\MyColorBox[backcolour]{\small\texttt{\vphantom{[]}#1}}}
\DeclareMathOperator{\im}{Im}
\DeclareMathOperator{\ri}{ri}
\DeclareMathOperator{\HF}{HF}
\DeclareMathOperator{\HP}{HP}
\DeclareMathOperator{\HS}{HS}
\DeclareMathOperator{\HN}{HN}
\DeclareMathOperator{\hn}{hn}
\DeclareMathOperator{\LT}{LT}
\DeclareMathOperator{\LM}{LM}
\DeclareMathOperator{\MS}{MS}
\DeclareMathOperator{\LC}{LC}
\DeclareMathOperator{\GL}{GL}
\DeclareMathOperator{\NR}{NR}
\DeclareMathOperator{\id}{id}
\DeclareMathOperator{\End}{End}
\DeclareMathOperator{\ord}{ord}
\DeclareMathOperator{\Ann}{Ann}
\DeclareMathOperator{\Ker}{Ker}
\DeclareMathOperator{\Hom}{Hom}
\DeclareMathOperator{\Rel}{Rel}
\DeclareMathOperator{\Mat}{Mat}
\DeclareMathOperator{\mult}{mult}
\DeclareMathOperator{\Supp}{Supp}
\DeclareMathOperator{\trace}{trace}
\DeclareMathOperator{\Char}{char} % \char already exists
\DeclareMathOperator{\tr}{tr} % transposed matrix
\DeclareMathOperator{\rank}{rk}
\DeclareMathOperator{\Lits}{Lits}
\DeclareMathOperator{\eval}{eval}
\DeclareMathOperator{\Pot}{\mathfrak{P}}
\DeclareMathOperator{\Nat}{\mathbb{N}}
\DeclareMathOperator{\Integ}{\mathbb{Z}}
\DeclareMathOperator{\Real}{\mathbb{R}}
\DeclareMathOperator{\Rat}{\mathbb{Q}}
\DeclareMathOperator{\Comp}{\mathbb{C}}
\DeclareMathOperator{\terms}{\mathbb{T}}
\DeclareMathOperator{\field}{\mathbb{F}}
\DeclareMathOperator{\calA}{\mathcal{A}}
\DeclareMathOperator{\calB}{\mathcal{B}}
\DeclareMathOperator{\calC}{\mathcal{C}}
\DeclareMathOperator{\calI}{\mathcal{I}}
\DeclareMathOperator{\calX}{\mathcal{X}}
\DeclareMathOperator{\calY}{\mathcal{Y}}
\DeclareMathOperator{\calZ}{\mathcal{Z}}
\newcommand{\lxor}{\mathop{\raisebox{0.5pt}{$\oplus$}}{}}
\DeclareMathOperator*{\LXOR}{\bigoplus}
\DeclareMathOperator*{\LAND}{\bigwedge}
\DeclareMathOperator*{\LOR}{\bigvee}
\newcommand{\liff}{\leftrightarrow}
\newcommand{\limplies}{\rightarrow}
\newcommand{\limpliedby}{\leftarrow}
\newcommand{\dotcup}{\mathop{\dot\cup}}
\newcommand{\true}{\mathcal{T}}
\newcommand{\True}{\true}
\newcommand{\false}{\mathcal{F}}
\newcommand{\False}{\false}
\newcommand{\Sols}{\mathcal{S}}
\newcommand{\Zeros}{\mathcal{Z}}
\newcommand{\cpp}{\textsf{C}\texttt{++}\xspace}
\newcommand{\bbar}[1]{{\mkern 4.0mu\overline{\mkern-4.0mu #1 \mkern-0.5mu}\mkern 0.5mu}}
\newcommand{\textmk}[1]{\textbf{#1}}  % marked expression e.g. in a definition
\newcommand{\textmkP}[1]{\textsl{#1}} % marked expression before its definition
\newcommand{\textmkI}[1]{\textsl{#1}} % marked expression during definition in introduction
\newcommand{\textmkN}[1]{\textsf{#1}} % for names like ApCoCoA or CrytoMiniSat
\newcommand{\code}[1]{\MyColorBox[backcolour]{\small\texttt{\vphantom{[]}#1}}}


% font size
\newcommand{\size}[1]{\fontsize{#1}{0}\selectfont{}}

% Inserts an empty page
\newcommand{\emptypage}{\newpage\null\thispagestyle{empty}\newpage}

% For restricting a function to a certain subset - f|_A
\newcommand\restr[2]{{% we make the whole thing an ordinary symbol
		\left.\kern-\nulldelimiterspace % automatically resize the bar with \right
		#1 % the function
		\vphantom{\big|} % pretend it's a little taller at normal size
		\right|_{#2} % this is the delimiter
}}

% Hyperlinks in TOC, references and citations
\hypersetup{
	colorlinks,
	citecolor=black,
	filecolor=black,
	linkcolor=black,
	urlcolor=black
}



%%%%%%%%%%%%%   MyColorBox for \code from StackExchange   %%%%%%%%%%%%%%
%Define a reference depth. 
%You can choose either relative or absolute.
%--------------------------
\newlength{\DepthReference}
\settodepth{\DepthReference}{g}%relative to a depth of a letter.
%\setlength{\DepthReference}{6pt}%absolute value.

%Define a reference Height. 
%You can choose either relative or absolute.
%--------------------------
\newlength{\HeightReference}
\settoheight{\HeightReference}{T}
%\setlength{\HeightReference}{6pt}


%--------------------------
\newlength{\Width}%

\newcommand{\MyColorBox}[2][red]%
{%
    \settowidth{\Width}{#2}%
    %\setlength{\fboxsep}{0pt}%
    \colorbox{#1}%
    {%      
        \raisebox{-\DepthReference}%
        {%
                \parbox[b][\HeightReference+\DepthReference][c]{\Width}{\centering#2}%
        }%
    }%
}